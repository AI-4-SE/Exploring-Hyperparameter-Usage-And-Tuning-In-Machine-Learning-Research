\documentclass[sigconf,review,anonymous]{acmart}

\acmConference[ESEC/FSE 2022]{The 30th ACM Joint European Software Engineering Conference and Symposium on the Foundations of Software Engineering}{14 - 18 November, 2022}{Singapore}

\overfullrule=2cm

\usepackage[utf8]{inputenc}
\usepackage[T1]{fontenc}

\usepackage[ruled, linesnumbered]{algorithm2e}

\usepackage{hyperref}

\usepackage{graphicx}
\usepackage{microtype}
\usepackage{balance}
\usepackage{booktabs}
\usepackage{multirow}
\renewcommand{\arraystretch}{1.2}
\usepackage{fancyvrb}
\usepackage{caption}
\DeclareCaptionType[placement={!ht}]{listing}[Listing][Code Listings]
\usepackage{wrapfig}
\usepackage[inline]{enumitem}
\usepackage{listings}

\newlist{questions}{enumerate}{2}
\setlist[questions,1]{label=RQ\textsubscript{\arabic*}:,ref=RQ\textsubscript{\arabic*}}

\graphicspath{{figures/}}


\title{Parameter Space in Machine Learning Projects}

\author{Sebastian Simon}
%\authornote{Both authors contributed equally to this research.}
%\authornotemark[1]
\affiliation{%
    \institution{Leipzig University}
    \country{Germany}
}

\renewcommand{\shortauthors}{Simon, et al.}

\clubpenalty=10000
\widowpenalty=10000
\displaywidowpenalty=10000

\begin{document}

\maketitle

\section{Introduction}\label{sec:intro}
% Motivation
\begin{itemize}
    \item machine learning has gained attention in recent years
    \item machine learning has become one of the most important research fields in academia as well as industrial research
    \item similarly to traditional software system, machine learning projects exhibit a wide range of configuration options
    \item those options influence, for instance, the performance of machine learning models, which features are used, or how data is processed
    \item especially algorithm-specific options are crucial for the models
    \item algorithm-specific options are introduced by machine learning libraries 
    \item the most popular machine learning libraries are sci-kit learn, tensorflow, pytorch
    \item those libraries provide ML algorithms and corresponding options for initialization
    \item option can be customized but often those libraries already provide default values
    \item default parameters of learning algorithm APIs may not be optimal for a given data or problem, and may lead to local optima (Paper: Code Smells for Machine Learning Applications)
    \item while the default parameters of a machine learning library may be adequate for some time, these default parameters may change in new versions of the library (Paper: Code Smells for Machine Learning Applications)
\end{itemize}

% Current Approaches/State of the Art
\begin{itemize}
    \item previous work already showed that configurations of machine learning may be a severe threats
    \item configurations part of technical debt in machine learning
    \item to the best of our knowledge, know studies that directly investigate the parameter space of machine learning libraries
\end{itemize}

% Own approach
\begin{itemize}
    \item we analyze projects that incorporate machine learning projects
    \item extracting their machine learning algorithms, such as classifier, regressor, or estimator, including all options and their values
    \item analyzing the data from two perspectives: project- and algorithm specific
\end{itemize}

% Research Questions
To guide our paper, we answer the following questions:
\begin{questions}
    \item How do researcher/developer/people approach machine learning? 
    \item Which domains, tasks, and datasets are tackled by ML scientists? 
    \item What are the differences between these domains, tasks, and datasets?    
    \item How is ML code configured? (project- or algorithm-specific) 
    \item What the parameter space of machine learning libraries?
    \item Which and what kind of values are used?
    \item What is the value range/parameter distribution of options?
    \item By which rule do people decide to use specific/default values?
\end{questions}

% Results

% Contribution
In summary, we make the following contributions:
\begin{itemize}
    \item A comprehensive study on the paperswithcode dataset.
    \item A comprehensive analysis of the parameter space in machine learning projects in terms of the most common machine learning libraries.
\end{itemize}


\section{State-of-the-Art}\label{sec:background}
- Introduction of ML's importance
- Which areas are affected?
- How are configuration options find?


\section{Methodology}\label{sec:methodology}

\subsection{Subject Systems}
\begin{itemize}
    \item repositories published along with scientific papers
    \item meta AI has instigated the the open source project \emph{Papers with Code}
    \item crawled this papers and their code to create a sample set of X repositories
    \item to ensure considering machine learning projects that incorporate the target library, we employed regular expression to check whether these libraries where used within the projects
    \item specifically, we focused here on import statements in the source code
    \item each python file was inspected and checked
    \item projects that fulfilled the condition has been added to the final set of subjects Systems
    \item this procedure was conducted for each library
\end{itemize}

\subsection{ML API Crawling}
\begin{itemize}
    \item for each machine learning library, we crawled the API
    \item thereby we extract all machine learning algorithms (classes) and their corresponding options used for initialization
    \item we dropped all methods, e.g., methods used for calculating metrics
    \item on top of that, we employ a data flow analysis to ensure receiving the correct option values
\end{itemize}

\subsection{Extracting Configuration Options}
\begin{itemize}
    \item static code analysis to locate occurrences of machine learning algorithm and to extract the options
    \item on top of that, we employ a data flow analysis to ensure that we get correct option values
\end{itemize}

\subsection{Analysis}

\section{Results}\label{sec:results}

\section{Discussion}\label{sec:discusssion}
\section{Threats-to-Validity}\label{sec:threats}
% External Validity
% Internal Validity
% Construct to Validity
\section{Conclusion}\label{sec:conclusion}

\begin{acks}
\end{acks}

\bibliographystyle{ACM-Reference-Format}
\bibliography{bibliography}
\end{document}